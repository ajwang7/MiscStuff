\documentclass[11pt,a4paper]{moderncv}
\moderncvtheme[orange]{classic}
\usepackage[utf8]{inputenc}
\usepackage[top=1.0cm, bottom=1.1cm, left=1.5cm, right=1.5cm]{geometry}
%\usepackage{hyperref}
\newcommand{\ts}{\textsuperscript}

% Largeur de la colonne pour les dates
\setlength{\hintscolumnwidth}{3.0cm}

\firstname{Matthieu}
\familyname{Riegler}
\title{Ingénieur de développement logiciel}
\photo[96pt][0.4pt]{photo2.jpg}
\address{44 Rue Bizanet}{38000 Grenoble}
\email{matthieu@riegler.fr}
\homepage{www.riegler.fr}
\mobile{06 16 50 07 20}
\extrainfo{23 ans -- Permis B}
%\social[twitter]{Kyro38}
%\quote{Expect the best, plan for the worst, and prepare to be surprised}


%%%%%%%%%%%%%%%%%%%%%%%%%%%%%%%%%%%%%%%%%%%%%%%%%%%%%%%%%%%%%%%%%%%%%%%%%%%%%%%%%%%%%%%%%%%

%%% redefining the space before the content below the title

\makeatletter
\renewcommand*{\makecvtitle}{%
  % recompute lengths (in case we are switching from letter to resume, or vice versa)
  \recomputecvlengths%
  % optional detailed information (pre-rendering)
  \def\phonesdetails{}%
  \collectionloop{phones}{% the key holds the phone type (=symbol command prefix), the item holds the number
    \protected@edef\phonesdetails{\phonesdetails\protect\makenewline\csname\collectionloopkey phonesymbol\endcsname\collectionloopitem}}%
  \def\socialsdetails{}%
  \collectionloop{socials}{% the key holds the social type (=symbol command prefix), the item holds the link
    \protected@edef\socialsdetails{\socialsdetails\protect\makenewline\csname\collectionloopkey socialsymbol\endcsname\collectionloopitem}}%
  \newbox{\makecvtitledetailsbox}%
  \savebox{\makecvtitledetailsbox}{%
    \addressfont\color{color2}%
    \begin{tabular}[b]{@{}r@{}}%
      \ifthenelse{\isundefined{\@addressstreet}}{}{\makenewline\addresssymbol\@addressstreet%
        \ifthenelse{\equal{\@addresscity}{}}{}{\makenewline\@addresscity}% if \addresstreet is defined, \addresscity and addresscountry will always be defined but could be empty
        \ifthenelse{\equal{\@addresscountry}{}}{}{\makenewline\@addresscountry}}%
      \phonesdetails% needs to be pre-rendered as loops and tabulars seem to conflict
      \ifthenelse{\isundefined{\@email}}{}{\makenewline\emailsymbol\emaillink{\@email}}%
      \ifthenelse{\isundefined{\@homepage}}{}{\makenewline\homepagesymbol\httplink{\@homepage}}%
      \socialsdetails% needs to be pre-rendered as loops and tabulars seem to conflict
      \ifthenelse{\isundefined{\@extrainfo}}{}{\makenewline\@extrainfo}%
    \end{tabular}
  }%
  % optional photo (pre-rendering)
  \newbox{\makecvtitlepicturebox}%
  \savebox{\makecvtitlepicturebox}{%
    \ifthenelse{\isundefined{\@photo}}%
    {}%
    {%
      \hspace*{\separatorcolumnwidth}%
      \color{color1}%
      \setlength{\fboxrule}{\@photoframewidth}%
      \ifdim\@photoframewidth=0pt%
        \setlength{\fboxsep}{0pt}\fi%
      \framebox{\includegraphics[width=\@photowidth]{\@photo}}}}%
  % name and title
  \newlength{\makecvtitledetailswidth}\settowidth{\makecvtitledetailswidth}{\usebox{\makecvtitledetailsbox}}%
  \newlength{\makecvtitlepicturewidth}\settowidth{\makecvtitlepicturewidth}{\usebox{\makecvtitlepicturebox}}%
  \ifthenelse{\lengthtest{\makecvtitlenamewidth=0pt}}% check for dummy value (equivalent to \ifdim\makecvtitlenamewidth=0pt)
    {\setlength{\makecvtitlenamewidth}{\textwidth-\makecvtitledetailswidth-\makecvtitlepicturewidth}}%
    {}%
  \begin{minipage}[b]{\makecvtitlenamewidth}%
    \namestyle{\@firstname\ \@lastname}%
    \ifthenelse{\equal{\@title}{}}{}{\\[1.25em]\titlestyle{\@title}}%
  \end{minipage}%
  \hfill%
  % optional detailed information (rendering)
  \llap{\usebox{\makecvtitledetailsbox}}% \llap is used to suppress the width of the box, allowing overlap if the value of makecvtitlenamewidth is forced
  % optional photo (rendering)
  % -------------------------------------------------------------------
  % this part immediately below has been modified for your purposes
  % -------------------------------------------------------------------
  \usebox{\makecvtitlepicturebox}\\[0.5em]%
  % optional quote
  \ifthenelse{\isundefined{\@quote}}%
    {}%
    {{\centering\begin{minipage}{\quotewidth}\centering\quotestyle{\@quote}\end{minipage}\\[2.5em]}}%
  \par}% to avoid weird spacing bug at the first section if no blank line is left after \makecvtitle
\makeatother
%%%%%%%%%%%%%%%%%%%%%%%%%%%%%%%%%%%%%%%%%%%%%%%%%%%%%%%%%%%%%%%%%%%%%%%%%%%%%%%%%%%%

\begin{document}
   \maketitle

   \section{Formation}
   \cventry{2011 -- 2013}{Master Mathématiques et Informatique}{Université Joseph Fourier, Grenoble (38)}{}{}{Filière Génie Informatique — spécialisation en IHM et technologies mobiles}
   \cventry{2007 -- 2011}{Licence d'informatique}{Université Joseph Fourier, Grenoble (38)}{}{}{}
   \cventry{2004 -- 2007}{Baccalauréat Scientifique}{Lycée International, Grenoble (38)}{}{}{Section allemande, obtention du diplome d'allemand KMK (équivalent C1)}

   \section{Expériences professionelles}
   \cventry{Avril 2013\\à Sept 2013\\(6 mois)} {Stage de fin d'études} {Orange Business Services / IT\&Labs, Montbonnot (38)} {}{} {
   \begin{itemize}
      \item {Stage R\&D pour le compte d'\emph{Orange Labs} dans le FabLab \emph{Thinging!}}
      \item {Reflexion menée sur l'Internet des objets et définition des projets à mener }
      \item {Réalisation d'une application mobile Android (Java)}
      \item {Réalisation d'une application serveur en Python avec le framework Django}
      \item {Gestion de projet en agilité/Méthode Scrum. Réalisation de la documentation de génie logiciel.}
   \end{itemize}}
   \cventry{Fév 2012\\à Juin 2012}{Stagiaire}{Laboratoire d'Informatique de Grenoble, Équipe IIHM}{Grenoble (38)}{}{
   \begin{itemize}
      \item{TER, sujet : « Techniques de sélection en réalité augmentée sur dispositif mobile » }
      \item{Développement d’une application de réalité augmentée sur iPhone/iPad (Objective-C, OpenGL).}
      \item{Réalité augmentée basée sur le Framework Qualcomm Vuforia (QCAR)}
   \end{itemize}}
   \cventry{Sept 2011\\à Juin 2013}{Tuteur référent} {Université Joseph Fourier}{Grenoble (38)}{} {Chargé de tutorat pour des 2\ts{e} année de Licence, d'écriture et de correction des examens de C2i.}
   \cventry{Juil 2008\\(1 mois)}{Ouvrier}{Südzucker}{Ochsenfurt (Allemagne)}{}{Ouvrier chargé d'approvisionnement en sucre.}

   \section{Compétences}
   \cvitem{Systèmes} {OSX, GNU/Linux (Ubuntu), Windows (XP, 7), iOS, Android.}
   \cvitem{Langages} {C/C++/Objective-C, Java/JEE, Python, scripting Bash, SQL, LaTeX, XML, CSS.}
   \cvitem{Conception} {UML, Design pattern, conception orientée objet, gestion de versions.}
   \cvitem{Logiciels} {Eclipse, Netbeans, Emacs, XCode, MS Office, Photoshop/Lightroom.}
   \cvitem{Frameworks} {Cocoa, Qt, Swing, OpenGL, Django.}
   \cvitem{IHM} {Multimodalité, Ergonomie des interfaces, Modèles de tâches, Vision par ordinateur.}
   \cvitem{Autre} {Gestion de projet, travail en équipe, méthodes agiles, techniques de tests, gestion de versions (svn, git), veille informatique.}
   \cvitem{Langues} {\textbf{Anglais} : Bon niveau. Lu, écrit, parlé.\newline{}%
   \textbf{Allemand} : Bilingue – lu, écrit et parlé couramment – Section allemande en Lycée International – séjours réguliers en Allemagne depuis l'enfance.}

   \section{Centres d'intérêt}
   \begin{itemize}
      \item {Photographie (portrait), cinéma en version originale (thriller, policier, comédie), bricolage.}
      \item {Cyclisme sur route, cyclotourisme (Francfort-Grenoble).}
      \item {Développement de logiciels et scripts. Voir \url{http://matthieu.riegler.fr/dev} }
      \item {Contributeur \& administrateur de la Wikipédia francophone depuis 2007. Organisation des ateliers Wikipédia de Grenoble en partenariat avec le CCSTI.}
   \end{itemize}
\end{document}
